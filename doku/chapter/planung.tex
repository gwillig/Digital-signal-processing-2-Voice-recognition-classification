\section{Planung und Vorbereitungen}

Die Bearbeitung der Projektaufgabe erfolgt in der Programmsprache \en{Python}, da die Gruppenmitglieder hier bereits einige Vorkenntnisse mitbringen. 
Ein weiterer Vorteil von Python ist dessen Plattformunabhängigkeit, wodurch das Projekt ohne Anpassungen sowohl auf Linux- als auch auf Windows-Betriebssystemen lauffähig ist.

Zum Trainieren von verschiedenen Modellen wurden zunächst die Wörter in der nachfolgenden Tabelle \ref{tab:recorded_words} erzeugt.
Für jedes Wort liegen insgesamt 54 Merkmalsvektorfolgen vor.
Dabei stammen jeweils 36 von zwei männlichen Sprechern und 18 von einer weiblichen Sprecherin. 
Um die Abhängigkeit der Klassifikationsqualität von der Umgebung zu vermindern, wurden die Aufnahmen zur Hälfte in einer ruhigen Umgebung (G117) und mit Hintergrundgeräuschen (A409, laufender Beamer) gesprochen. 
Weiterhin wurde eine Aufnahme der im jeweiligen Raum herrschenden \en{Stille} erzeugt.


\bigskip
{\setlength{\extrarowheight}{2pt}%
\begin{table}[!hp]
\centering
\begin{tabular}{| c | l |}
\hline
Namen		 & Andreas, Gustav, Lisa  \\ \hline
Zahlen		 & Zwei, Drei               \\ \hline
Hausaufgaben & Signalverarbeitung, Fouriertransformation \\ \hline
sonstige 	 & mögen, und 	\\ \hline
\end{tabular}
\caption[Erzeugte Merkmalsvektorfolgen]{Erzeugte Merkmalsvektorfolgen}
\label{tab:recorded_words}
\end{table}
\bigskip



\begin{comment}
Die Berechnung der Distanzen kann durch verschiedene Methoden erfolgen. Folgenden Methoden wurden verwendet:

\begin{itemize}
	\item Euklidischer Abstand
	\item Mahalanobis Abstand
	\item Abstandsmaß basierend auf der Normalverteilung
	\item HMM (als Produktfolge)
	\item HMM (Maximum Approximation)
	\item HMM (negativ logarithmierte Wahrscheinlichkeiten)
\end{itemize}
\end{comment}